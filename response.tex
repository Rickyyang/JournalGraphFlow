%%%%%%%%%%%%%%%%%%%%%%%%%%%%%%%%%%%%%%%%%
% Academic Paper Response Letter. 
% LaTeX Template
% Version 1.0 (20/07/2023)
% 
% This template refers to the toolbox tutorial (in Chinese):
% https://liam.page/2016/07/22/using-the-tcolorbox-package-to-create-a-new-theorem-environment/
% Please refer to the tutorial for your customization. 
% 
% Template license: Creative Commons CC BY 4.0
%
% Author: Yuhan Xie
% Contact email: yhxie0107@gmail.com
% Any revision advice is welcome!
% 
%%%%%%%%%%%%%%%%%%%%%%%%%%%%%%%%%%%%%%%%%


\documentclass{article}
%% Language and font encodings
\input{preamble/common}
\usepackage[english]{babel}
\usepackage{times} % make the section title in Times New Roman

\usepackage{booktabs}
\usepackage{tabu}
\usepackage[T1]{fontenc}

%% Sets page size and margins
\usepackage[a4paper,top=2.5cm,bottom=2cm,left=1.7cm,right=1.7cm,marginparwidth=1.75cm]{geometry}

%% Useful packages
\usepackage{amsmath}
\usepackage{graphicx}
\usepackage[colorinlistoftodos]{todonotes}
% \usepackage[colorlinks=true, allcolors=blue]{hyperref}
% \usepackage{indentfirst}
% \usepackage{graphicx}
% \usepackage{subfigure}
\usepackage{float}
\usepackage{threeparttable} 
\usepackage{multirow}
\usepackage{url}


\usepackage[most]{tcolorbox}
\newcommand{\rrtstar}{$\texttt{RRT}^\texttt{*}$}
\newcommand{\SubItem}[1]{
    {\setlength\itemindent{15pt} \item[-] #1}
}

\newtheorem{constraint}{ADMM constraint}
\newtheorem{example}{Example}

\usepackage[export]{adjustbox} 
\usepackage[capitalise]{cleveref}
\usepackage{xcolor}
\newcommand{\new}[1]{\textcolor{blue}{#1}}
\newcommand{\news}{\color{blue}}
\graphicspath{{Figs/}}



\begin{document}
	

\newtcbtheorem[auto counter, number within = section]{cmt}{Comment}{
	colbacktitle = black!60!white, colframe = black!60!white,
	colback = black!5!white,
	fonttitle=\bfseries,%fontupper=\itshape,
}{t}



\noindent
Dear Editor and Reviewers:

\vspace{0.2cm}
\noindent
First and foremost, we sincerely thank the Editor-in-Chief, Associate Editor, and anonymous reviewers for their insightful comments and suggestions. These have been carefully incorporated into the revised manuscript, resulting in a clearer, more informative, and more readable version. We acknowledge the reviewers' concerns regarding grammatical errors, unclear definitions, and poorly formatted figures, which were noted as obstacles to evaluating the paper's technical contributions. In this revision, we have thoroughly addressed these issues by eliminating grammatical errors, refining definitions and notations for clarity, and replotting the figures to enhance readability. All reviewer comments have been carefully addressed, and the corresponding changes have been highlighted in \new{blue} for ease of reference. Our detailed responses to the reviewers' comments, provided in a 1:1 correspondence, are outlined below.
\vspace{0.2cm}
\noindent
Yours sincerely,

%\vspace{0.2cm}
\noindent
Ziqi Yang, Roberto Tron


%%%%%%%%%%%%%%%%%%%%%%%%%%%%% Rewiewer 1
\newpage
\section{Response to reviewer 1} 

%%%%%%%%%%%%% Comment 1.1
\begin{cmt*}{}{}

	This paper proposes a multi-agent mobile robot planning algorithm that
	is robust agains attackers that would compromise the agents by
	directing them to unsafe regions. This is achieved by incorporating 1)
	co-observation constraints so that the agents could watch over each
	other, 2) reachability constraints so that the region the agents could
	potentially enter does no overlap with unsafe regions, and 3) sub-team
	planning algorithm to improve robustness by adding more number of
	agents.
	
	The proposed method is interesting, but I think the writing must be
	improved significantly to secure the clarity, and there are a lot of
	grammar mistakes that need to be fixed. To me, it appears that the
	draft may have been prepared somewhat hastily, which is not considerate
	of the reviewers' time.

\end{cmt*}
\vspace{0.1cm}
\noindent
\underline{\textbf{Response:}}
\vspace{0.2cm}
We are grateful for your thoughtful comments and for expressing interest in our proposed method. We sincerely apologize for the grammatical mistakes in the initial submission and acknowledge the importance of presenting our work with clarity and professionalism. In the revised version, we have been extra careful to address all grammar issues and have implemented thorough proofreading to minimize similar mistakes in future submissions. We greatly appreciate your patience and constructive feedback, which have been invaluable in improving the quality of our manuscript.

\vspace{0.3cm}


%%%%%%%%%%%%% Comment 1.0
\begin{cmt}{}{}        %% use * when do not need enumerate
	It seems like Example 1 is introduced as an example to help
	readers understand the formulation (1) in details. However, it is still
	vague, and it doesn't easily connect to Figure 3b.
\end{cmt}
\vspace{0.1cm}
\noindent
\underline{\textbf{Response:}}
\vspace{0.2cm}

\begin{example}\label{example:map_exploration}
	\new{Robots are tasked to navigate an unknown task space, collecting sensory data to reconstruct a field (see Fig. 2a). The space is modeled as grid points, each with an associated value. The goal is to find paths that minimize uncertainty in reconstructing the field. Each grid point $j$ is associated with a Kalman Filter (KF) [28] to track uncertainty through its covariance $P_j$, updated based on the measurements taken by robots along the trajectory $\vq$, where measurement quality, modeled by a Gaussian radial basis function, is higher near the robot. The optimization objective $\varPhi(\vq)=\max_j P_j(\vq)$ is to minimize the maximum uncertainty (detailed in [6]).}
\end{example}
\begin{figure}[h]
	\centering
	\subfloat[\new{Map exploration task.}\label{fig:illustration}]{\includegraphics[width = 0.45\linewidth]{illustration figure_1}} 
	\subfloat[\new{Grid world result.}\label{fig:Grid-example-application}]{\includegraphics[width = 0.45\linewidth]{Grid_world}}
	\caption{\new{(a) Grid world example of a task planned in an $8 \times 8$ grid world. Blue grids are obstacles, orange and green grids are forbidden regions.}}
  \end{figure}
\vspace{0.3cm}


%%%%%%%%%%%%% Comment 1.2
\begin{cmt}{}{}
	Personally, I think to much space is dedicated to the details of how
	reachability constraints are implemented in the ADMM formulation (Sec.
	D, E, F, G, H). Although they are important details, I think this part
	can be condensed more, and some of the details can be deferred to the
	Appendix, so that some space can be dedicated more to improve the
	clarity of the other sections.
\end{cmt}
\vspace{0.1cm}
\noindent
\underline{\textbf{Response:}}
\vspace{0.2cm}


\vspace{0.3cm}


%%%%%%%%%%%%% Comment 1.3
\begin{cmt}{}{}

Comments 1.3

\end{cmt}
\vspace{0.1cm}
\noindent
\underline{\textbf{Response:}}
\vspace{0.2cm}

Your Response 1.3

\vspace{0.3cm}



%%%%%%%%%%%%%%%%%%%%%%%%%%%%% Rewiewer 2
\newpage
\section{Response to reviewer 2}

%%%%%%%%%%%%% Comment 2.1
\begin{cmt}{}{}

Comments 2.1

\end{cmt}
\vspace{0.1cm}
\noindent
\underline{\textbf{Response:}}
\vspace{0.2cm}

Your Response 2.1

\vspace{0.3cm}


%%%%%%%%%%%%% Comment 2.2
\begin{cmt}{}{}

Comments 2.2

\end{cmt}
\vspace{0.1cm}
\noindent
\underline{\textbf{Response:}}
\vspace{0.2cm}

Your Response 2.2

\vspace{0.3cm}


%%%%%%%%%%%%% Comment 2.3
\begin{cmt}{}{}

Comments 2.3

\end{cmt}
\vspace{0.1cm}
\noindent
\underline{\textbf{Response:}}
\vspace{0.2cm}

Your Response 2.3

\vspace{0.3cm}


%%%%%%%%%%%%% Comment 2.4
\begin{cmt}{}{}

Comments 2.4

\end{cmt}
\vspace{0.1cm}
\noindent
\underline{\textbf{Response:}}
\vspace{0.2cm}

Your Response 2.4

\vspace{0.3cm}


\end{document}